\section{Prevalence of Fingerprinting Cloaking}
\label{s:prevalence}

\begin{figure}
\centering
\includegraphics[width=\linewidth]{figs/server_cloaking2.png}
\caption{Simplified PHP code snippet of fingerprinting cloaking in a phishing kit, checking IP, Hostname, and User-Agent.}
\label{fig:servercloaking}
% \vspace{-10pt}
\end{figure}

To understand the prevalence of cloaking techniques used in advanced phishing kits, we manually inspect 56 phishing kits, and extract common patterns that implement fingerprinting cloaking techniques in phishing kits.

The patterns include (1) sensitive file names such as ``antibot'', ``blocker'', and ``antirobot''; (2) blocked words, for example, an array containing crawler information such as ``google'', ``paypal'', or ``phishtank'' (e.g., \texttt{blocked\_words} in~\autoref{fig:servercloaking}); (3) IP checker that blocks visit from certain IP addresses, such as \texttt{bannedIP} in~\autoref{fig:servercloaking}; and (4) error header, returning an error status code and an error web page, such as \texttt{header} function in PHP.
These attributes reflect the implementation of fingerprinting cloaking techniques in phishing kits.

With these patterns, we can automatically inspect such evasion to evaluate its prevalence.
Among the inspected kits in~\emph{phishunt.io}~\cite{phishunt} from January to June 2021, 198 of 236 contain cloaking techniques to fingerprint anti-phishing crawler visit through IP, Referrer, and/or User-Agent.
To comprehensively measure the prevalence of server-side fingerprinting cloaking techniques in phishing kits,
we also run our pattern finding script on Cisco's dataset that is contributed to Virus Total.
In total, all of 2,421 phishing kits from Cisco's dataset contain such evasion.
Among them, 1,983 of the advanced phishing kits contain User-Agent checker.
% Preliminarily, 100 out of 100 phishing kits from Cisco's dataset contain such evasion.
% , while only one implements server-side click-through cloaking.

From this experiment,
we show that phishers prefer to fingerprint visit through information in HTTP request to distinguish humans out of crawlers.
And thanks to the fact, 
% \spartacus 
we can design a framework to evade phishing websites generated from such advanced phishing kits by triggering the cloaking techniques.