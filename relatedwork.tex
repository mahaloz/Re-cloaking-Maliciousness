\section{Related Work}

Researchers have been studying phishing for several decades.
They have proposed several methodologies to detect phishing attacks based on one or some of features from URL, content, and etc, and then warn users before visiting the deceptive websites.
Some work analyzes URL of a suspicious website based on the lexical features or URL ranking to determine the maliciousness of the site~\cite{blum2010lexical, le2011phishdef, khonji2012enhancing, feroz2015phishing}.
Others collect web page content and detect phishing websites with textual and visual similarity features~\cite{zhang2007cantina, zhang2011textual, dunlop2010goldphish}.
Combining with available features including both URL and web page content, researchers have developed blacklist based anti-phishing systems such as Google Safe Browsing~\cite{whittaker2010large} to protect Internet users from visiting suspicious websites.
With the large-scale implementation of cloaking techniques in phishing attacks~\cite{oest2020sunrise, oest2020phishtime, oest2019phishfarm, oest2018inside}, researchers realize that the sophisticated phishing attacks are responsible for substantial portion of damage and that the whole ecosystem should prioritize on mitigating phishing with evasion techniques.
Therefore, analysis and detection of server-side~\cite{wang2011cloak, invernizzi2016cloak} and client-side~\cite{zhang2021crawlphish} cloaking techniques have been proposed to fight against such sophistications.

All the proposed methodologies, however, have a limitation that they only provide a pain-relief pills to certain symptoms of phishing attacks, such as suspicious URL, similar layout as the impersonated website, and cloaking techniques.
Furthermore, as phishers have been developing advanced cloaking techniques, it becomes more and more difficult to retrieve phishing content, which most anti-phishing systems depend on.
With very limited amount of features of a suspicious website, the system cannot determine precisely the maliciousness.
Therefore, with the contribution of previous studies on phishing symptoms, our work provides a vaccine to neutralizing advanced phishing-virus.
Our system is also extendable and always up-to-date by adding fingerprints that researchers will find in the future.

