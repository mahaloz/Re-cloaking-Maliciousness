\section{Introduction}

The security community has made efforts researching on phishing attacks,
but attackers still make profits using phishing websites and continuously harm the victims they target and the organizations they mimic~\cite{ho2019detecting, van2019cognitive}.
According to Google Transparency report, phishing attacks have replaced online malware to be the most prevalent web-based threat~\cite{googletransparencyreport, solutions2019verizon}.
Nowadays, phishing websites continue to grow in sophistication and hence can slip past modern defensive methodologies.
Therefore, the current anti-phishing ecosystem leaves advanced phishing websites ``golden hours'' to damage the whole Internet community~\cite{oest2020sunrise}.

Phishing websites adopt \emph{evasion} techniques to delay or evade detection by automated anti-phishing systems in the cat-and-mouse game,
which, in turn, maximize phishers' return-on-investment~\cite{thomas2017data}.
The evasion techniques, also known as~\emph{cloaking}, implemented in phishing websites, attempt to distinguish the visits from potential victims out of those from anti-phishing crawlers.
They will show real phishing content to visitors who they decide as ``real human'', while display a benign-looking web page to those who are identified as ``anti-phishing crawler''.
The damage brought by these phishing websites' efforts is not only that they steal just account numbers and passwords,
but that the phishing websites nowadays try to dump all information including victim's identity~\cite{thomas2017data}.
Thus, the cloaked phishing websites cause a wider damage to the whole society and are very difficult to effectively and efficiently mitigate~\cite{oest2020sunrise, zhang2021crawlphish}.

Thwarting phishers' evasion efforts is, thus, treated by the current anti-phishing ecosystem as a very important issue,
because they think that correct web page retrieval and timely detection is the key to successful mitigation~\cite{oest2020sunrise, zhang2021crawlphish}.
Following this trajectory, prior research has proposed methodologies categorizing and mitigating cloaking techniques in phishing~\cite{oest2018inside, oest2019phishfarm, zhang2021crawlphish}.
However, the server-side cloaking techniques can still defeat key ecosystem defenses such as blacklist-based mechanism~\cite{oest2019phishfarm}.
With so many filtering conditions in the cloaking techniques,
it is very difficult for the content-based anti-phishing systems to acquire real phishing content~\cite{oest2018inside, oest2020phishtime}.
This analysis magnifies an issue that current anti-phishing systems cannot provide a reliable protection for users against phishing websites with cloaking techniques.
So how about we consider the problem from a different angle:
the ultimate goal to mitigate or defeat phishing is to make potential victims not see any phishing content.
So why do we devote ourselves trying to sneak through all the challenges cloaking techniques set and reach to the phishing content?
Why not trigger the cloaking in user's browsers and let phishing websites return a benign-looking web page?
In this way, users will not see phishing content in real time, and a lose-lose situation is shown to phishers because they cannot differentiate visits from between users and anti-phishing crawlers.

To this end, we propose \spartacus, a framework that disguise Internet users as anti-phishing crawlers to request web page content of a suspicious URL, while remain users' own profiles on visiting benign websites.
Before visiting a URL, \spartacus queries the domain information, such as reputation, to decide whether to mutate the HTTP profile.
If not, \spartacus allows the request sent with user's default profile.
Otherwise, \spartacus mutates the items in the HTTP profile that will be inspected by cloaking techniques in the phishing server before sending requests to suspicious URLs.
When the cloaking script examines the HTTP request, it will identify that the visit is from an anti-phishing infrastructure, and will return a benign-looking web page content to users.

