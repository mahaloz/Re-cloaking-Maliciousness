\section{Conclusion}

Through the analysis of compromised phishing kits, we gain an understanding that fingerprinting cloaking techniques are largely implemented in the sophisticated phishing attacks on the server-side, and helping evade visits from crawler-like visits.
Such evasion is difficult to mitigate because phishers can always include new features of the up-to-date anti-phishing crawlers and identify them.

We consider this problem from a totally different perspective.
Instead of racking our brains describing the fingerprints phishing kits contain and designing new crawlers without those fingerprints, and learnt by phishers and then adding new fingerprints in, which is a dead loop,
why not simply trigger the cloaking from the client-side so that no user will see phishing content?
The \spartacus system addresses the new angle for the anti-phishing ecosystem to fight against cloaked phishing websites.
Without bothering getting around cloaking techniques, \spartacus beats the advanced phishing websites in their own game:
cloak users against cloaked phishing websites.
For benign websites, \spartacus is proved that it impacts negligibily on their access, layout, and functionalities to users.

Due to the rise of sophisticated phishing websites in the wild, we believe that automated evasion systems such as \spartacus are essential
to keep trapping phishers in a lose-lose dilemma where they cannot differentiate real user and anti-phishing crawler visit.
Methodology such as ours can be incorporated by the ecosystem to more expeditiously and more reliably evade sophisticated phishing right from the client side, 
which, in turn, can help prevent users from falling victim to these attacks through the
continuous camouflage to malicious evasion techniques.