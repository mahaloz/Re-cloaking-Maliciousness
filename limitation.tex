\section{Countermeasure of \spartacus}


There are different types of phishing websites in the wild, including basic and advanced.
Within advanced phishing websites, server-side and client-side cloaking techniques are two that help evade detection from the anti-phishing ecosystem.
Because \spartacus focuses on the evasion of phishing attacks with server-side cloaking techniques,
there may be a countermeasure that attackers can use other types of phishing websites to harm individuals and organizations.

Phishers can use basic phishing websites, phishing websites with client-side cloaking, or those with only IP/Hostname filtering.
As we discussed above and analysis result from Oest et al.~\cite{oest2020phishtime}, basic phishing websites can be detected and blacklist by anti-phishing systems as fast as 28 minutes.
Distributing basic phishing websites cannot help phishers maximize their return-on-investment.
As for client-side cloaking techniques,
phisher can implement them into their websites to bypass \spartacus.
However, Zhang et al.~\cite{zhang2021crawlphish} has proposed a methodology to detect such evasion by force-executing JavaScript.
Hence, client-side cloaked phishing websites can be timely detected as well.
At last, phishers can loose the evasion criteria, such as by only checking IP and Hostnames.
Anti-phishing infrastructures can learn the lesson and set up the anti-phishing crawlers in residential IPs.
Such move is equal to welcome crawlers to visit.

All in all, with \spartacus deployed and current support from the anti-phishing ecosystem, 
it is very difficult to bypass all anti-phishing methodologies and allow only potential victim traffic in at the same time.
Such dilemma forces attackers to either spend resources inventing new evasion techniques, or quit phishing due to little profit.


\section{Limitation}

Even though \spartacus can evade a diverse array of sophisticated phishing websites using server-side cloaking techniques in the wild,
our framework should be considered alongside certain limitations.


\subsection{\spartacus Design}

\noindent
\textbf{Phishing Classification.}
The \spartacus framework is not a phishing classification system.
Rather, it camouflages users as security crawlers when they visit suspicious websites with cloaking techniques and can evade malicious content if they contain.
The reason why \spartacus cannot classify its maliciousness is because it does not need to.
As evaluated in~\autoref{s:eval}, \spartacus can evade 82.28\% of phishing websites in real time, while has a negligible impact on benign websites.
Previous work has proposed methodologies classifying phishing websites with high accuracy~\cite{whittaker2010large, lin2021phishpedia}.
Therefore, with \spartacus and existing methodologies, the anti-phishing ecosystem can cover a wider range of phishing attacks.


\noindent
\textbf{HTTP request mutation.}
As discussed in~\autoref{s:background}, fingerprinting cloaking techniques in phishing server can inspect IP, Hostname, User-Agent, and Referrer to classify whether the visitor is an anti-phishing crawler.
In \spartacus design, we only consider to mutate User-Agent and Referrer in the HTTP request.
This is because without permission from server owners, we cannot use their IP and Hostname to route our HTTP request.
So, it may contain a limitation where \spartacus cannot evade phishing websites that only identify crawlers/bots by IPs and Hostnames.
However, with our analysis on phishing kits, we found that all phishing kits with fingerprinting cloaking techniques implemented IP, User-Agent, Referrer, and Hostnames.
This is because phishers want to block every suspicious visit that may be from anti-phishing infrastructures.
And hence the attackers accumulate many criteria to evade crawlers.
For example, there are 407 sensitive words phishing kits check to deny request, and one kit blocks as many as 2,827,521 IPs using regular expression.
% 2,818,048 + 9,472 + 1
Therefore, we believe that mutating User-Agent and Referrer is adequate for \spartacus to evade malicious content.

\subsection{\spartacus Evaluation}

\noindent
\textbf{Phishing kit analysis.}
In the analysis to understand the prevalence of fingerprinting cloaking, we hope to include as many phishing kits as possible to have a comprehensive analysis.
Due to the resource limitation, we only did our analysis on phishing kits from \emph{phishunt.io}~\cite{phishunt} and those from public dataset from Cisco.
Within both resources, we were able to analyze XYZ phishing kits and summarized that the fingerprinting cloaking techniques exist in XX\% of the phishing kits.
We believe that such analysis can describe its prevalence preliminarily to indicate the potential usage of our \spartacus framework.


\noindent
\textbf{Data collection.}
We select APWG dataset to evaluate the effectiveness of \spartacus.
And due to infrastructure and resource limitations, we were only able to test \spartacus over a total of nine months from November 2020 to July 2021.
Even though additional data crawling would be desirable to evaluate \spartacus,
the APWG dataset can provide the breadth of phishing data collection because it contains different types of phishing websites targeting different brands, which are submitted periodically by collaborating members including anti-phishing systems and financial organizations impersonated by phishing websites.
Besides, we have tested \spartacus on over 130,000 live phishing websites and verified that it can evade malicious content.
Therefore, we believe such limitation can be mitigated to some extent.

\noindent
\textbf{}