\section{Evading Server-side Cloaked Phishing}

According to our observation and analysis in~\Cref{s:eval}, phishers take advantage of server-side cloaking techniques in their attacks.
The exclusive feature of such evasion is to inspect the HTTP profile of the visitor and to respond with the corresponding content,
the algorithm of which, still, remains behind the scene.
Phishers adopt the strategy because they believe that it is difficult for automated anti-phishing systems to depict their evasion algorithms.
Even though one phishing website can be detected and blacklisted, they can re-generate another and continue harming the whole anti-phishing ecosystem.
Phishers decide that rather than blocking all suspicious visits that may contain ones from potential victims, they will not allow one from anti-phishing systems in.
We expect that the sophistication of phishing website will continue to grow and the advanced phishing kits will set more criteria to inspect, match, and block the suspicious in-coming request.
Although researchers and organizations have proposed mitigations phishing websites~\cite{xiang2011cantina+, lin2021phishpedia}, they all require malicious web page content to feed the classifier and make the decision.
Server-side cloaking techniques deny the requests from the anti-phishing systems and their methodologies may not be useful.
We select 500 cloaked phishing websites that can evaded by \spartacus and 500 benign websites from Alexa Top One Million list to test CANTINA+~\cite{xiang2011cantina+}, which is a phishing classifier with URL, web, and HTML based features.
The result in~\autoref{tab:newmleval} shows that it has a high false-negative rate when classifying cloaked phishing websites.
Therefore, the ecosystem should ensure
that existing and new detection and mitigation systems are capable of adapting to such evasion techniques.

\newmleval

To mitigate phishing websites with server-side cloaking techniques, anti-phishing crawlers should carefully modify their HTTP requests to pretend as much alike human visitors as possible.
In this case, the crawlers can bypass the cloaking techniques and hence can retrieve the real malicious content.
For example, they should avoid sending requests based on the IP addresses of well-known anti-phishing entities.
Besides, crawler- and famous-brand- looking strings should not be used in User-Agent or host name.
In addition, a suspicious website returning a benign-looking web page to anti-phishing crawlers, does not mean that this website should be marked as benign.
Instead, crawlers should attempt to visit the website with different HTTP profiles before making the decision.

Also, to protect Internet users from advanced phishing websites timely and trivially, \spartacus can be adopted into the browser directly.
Without cumbersome procedures such as submitting suspicious URLs, retrieving correct web page content, classifying maliciousness, and notifying users,
\spartacus can evade malicious content on phishing websites with fingerprinting cloaking techniques.
Users will not differentiate between a default browser and a \spartacus'ed one, according to our evaluation in~\Cref{s:eval}.
For phishers, they have the similar feeling, but they cannot differentiate the HTTP requests from real visitors and anti-phishing crawlers.
Therefore, \spartacus would be particularly helpful in instantly evading phishing websites that implement cloaking techniques with high confidence.