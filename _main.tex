%%%%%%%%%%%%%%%%%%%%%%%%%%%%%%%%%%%%%%%%%%%%%%%%%%%%%%%%%%%%%%%%%%%%%%%%%%%%%%%%
% Template for USENIX papers.
%
% History:
%
% - TEMPLATE for Usenix papers, specifically to meet requirements of
%   USENIX '05. originally a template for producing IEEE-format
%   articles using LaTeX. written by Matthew Ward, CS Department,
%   Worcester Polytechnic Institute. adapted by David Beazley for his
%   excellent SWIG paper in Proceedings, Tcl 96. turned into a
%   smartass generic template by De Clarke, with thanks to both the
%   above pioneers. Use at your own risk. Complaints to /dev/null.
%   Make it two column with no page numbering, default is 10 point.
%
% - Munged by Fred Douglis <douglis@research.att.com> 10/97 to
%   separate the .sty file from the LaTeX source template, so that
%   people can more easily include the .sty file into an existing
%   document. Also changed to more closely follow the style guidelines
%   as represented by the Word sample file.
%
% - Note that since 2010, USENIX does not require endnotes. If you
%   want foot of page notes, don't include the endnotes package in the
%   usepackage command, below.
% - This version uses the latex2e styles, not the very ancient 2.09
%   stuff.
%
% - Updated July 2018: Text block size changed from 6.5" to 7"
%
% - Updated Dec 2018 for ATC'19:
%
%   * Revised text to pass HotCRP's auto-formatting check, with
%     hotcrp.settings.submission_form.body_font_size=10pt, and
%     hotcrp.settings.submission_form.line_height=12pt
%
%   * Switched from \endnote-s to \footnote-s to match Usenix's policy.
%
%   * \section* => \begin{abstract} ... \end{abstract}
%
%   * Make template self-contained in terms of bibtex entires, to allow
%     this file to be compiled. (And changing refs style to 'plain'.)
%
%   * Make template self-contained in terms of figures, to
%     allow this file to be compiled. 
%
%   * Added packages for hyperref, embedding fonts, and improving
%     appearance.
%   
%   * Removed outdated text.
%
%%%%%%%%%%%%%%%%%%%%%%%%%%%%%%%%%%%%%%%%%%%%%%%%%%%%%%%%%%%%%%%%%%%%%%%%%%%%%%%%

\documentclass[letterpaper,twocolumn,10pt]{article}
\usepackage{usenix2019_v3}

% to be able to draw some self-contained figs
\usepackage{tikz}
\usepackage{amsmath}
\usepackage{comment}
\usepackage{pdfcomment}
\usepackage{color}
\usepackage{graphicx}
\usepackage{caption}
\usepackage{subcaption}
\usepackage{threeparttable}
\usepackage{multirow}
\usepackage{booktabs}
\usepackage{verbatim}
\usepackage{epstopdf}
\usepackage{rotating}
\usepackage{listings}
\usepackage{listing}
\usepackage{paralist}
\usepackage{arydshln}
\let\labelindent\relax
\usepackage{enumitem,amssymb}
\usepackage{algpseudocode}% http://ctan.org/pkg/algorithmicx
\usepackage{balance}
\usepackage{endnotes}
\usepackage{xspace}
\usepackage{times}
\usepackage{amsfonts}
\usepackage{pifont}
\usepackage{wasysym}
\usepackage{tikz}
\usepackage{hyperref}
\usepackage{minted}
\usepackage{xcolor}
\usepackage{colortbl}
\usepackage{soul}
\usepackage[normalem]{ulem}
\usepackage{algpseudocode}
\usepackage{algorithm}
\usepackage{cleveref}
% \usepackage[noadjust]{cite}
% \renewcommand{\citepunct}{,\penalty\citepunctpenalty\,}
% \renewcommand{\citedash}{--}

\newcommand{\spartacus}{{Spartacus}\xspace}

% inlined bib file
% \usepackage{filecontents}

% %-------------------------------------------------------------------------------
% \begin{filecontents}{\jobname.bib}
% %-------------------------------------------------------------------------------

% \end{filecontents}

%-------------------------------------------------------------------------------
\begin{document}
%-------------------------------------------------------------------------------

%don't want date printed
\date{}

% make title bold and 14 pt font (Latex default is non-bold, 16 pt)
\title{Spartacus: Cloak Users Against Cloaked Phishing Websites}

%for single author (just remove % characters)
% \author{
% {\rm Your N.\ Here}\\
% Your Institution
% \and
% {\rm Second Name}\\
% Second Institution
% % copy the following lines to add more authors
% % \and
% % {\rm Name}\\
% %Name Institution
% } % end author

\maketitle

%-------------------------------------------------------------------------------
\begin{abstract}
%-------------------------------------------------------------------------------
% Your abstract text goes here. Just a few facts. Whet our appetites.
% Not more than 200 words, if possible, and preferably closer to 150.
Phishing has been the top online attack nowadays.
Phishers implement cloaking techniques to evade detection from anti-phishing systems by checking profiles from HTTP requests in server side as well as from browsers in client side.
The anti-phishing ecosystem has been devoting to reveal real phishing content behind cloaks.
However, it has few progress because phishers can always enrich their fingerprinting database by adding new rules to distinguish traffics from the ecosystem and hence evade the visit.

According to the situation where cloaking techniques are widely implemented in phishing attacks and the goal of anti-phishing is to prevent Internet users to see phishing contents, we consider the security problem from a different perspective: instead of trying best to reveal phishing content, we can leverage the cloaking techniques in phishing to prevent users to see phishing content. In this work, we propose \emph{Spartacus}, a framework that disguise Internet users as anti-phishing crawlers to request web page content of a suspicious URL, while remain users' own profiles on visiting whitelisted websites.

\textcolor{blue}{Evaluation plan:}
We evaluated Spartacus from different aspects: (1) Effectiveness: we have two groups open reported phishing websites, one with normal browser, the other with ``Spartacused'' browser, for three months. We expect to show Group 1 saw way more phishing than Group 2.
(2) Efficiency: Two groups visit same amount of websites to test the latency of Spartacus.
Also we use Spartacus to visit legitimate websites to evaluate the impact (layout of the websites).
Then, we evaluate how difficult is to maintain our Spartacus system.
(3) adversarial mindset
(4) other fingerprints to get

\end{abstract}

\section{Introduction}

The security community has made efforts researching on phishing attacks,
but attackers still make profits using phishing websites and continuously harm the victims they target and the organizations they mimic~\cite{ho2019detecting, van2019cognitive}.
According to Google Transparency report, phishing attacks have replaced online malware to be the most prevalent web-based threat~\cite{googletransparencyreport, solutions2019verizon}.
Nowadays, phishing websites continue to grow in sophistication and hence can slip past modern defensive methodologies.
Therefore, the current anti-phishing ecosystem leaves advanced phishing websites ``golden hours'' to damage the whole Internet community~\cite{oest2020sunrise}.

Phishing websites adopt \emph{evasion} techniques to delay or evade detection by automated anti-phishing systems in the cat-and-mouse game,
which, in turn, maximize phishers' return-on-investment~\cite{thomas2017data}.
The evasion techniques, also known as~\emph{cloaking}, implemented in phishing websites, attempt to distinguish the visits from potential victims out of those from anti-phishing crawlers.
They will show real phishing content to visitors who they decide as ``real human'', while display a benign-looking web page to those who are identified as ``anti-phishing crawler''.
The damage brought by these phishing websites' efforts is not only that they steal just account numbers and passwords,
but that the phishing websites nowadays try to dump all information including victim's identity~\cite{thomas2017data}.
Thus, the cloaked phishing websites cause a wider damage to the whole society and are very difficult to effectively and efficiently mitigate~\cite{oest2020sunrise, zhang2021crawlphish}.

Thwarting phishers' evasion efforts is, thus, treated by the current anti-phishing ecosystem as a very important issue,
because they think that correct web page retrieval and timely detection is the key to successful mitigation~\cite{oest2020sunrise, zhang2021crawlphish}.
Following this trajectory, prior research has proposed methodologies categorizing and mitigating cloaking techniques in phishing~\cite{oest2018inside, oest2019phishfarm, zhang2021crawlphish}.
However, the server-side cloaking techniques can still defeat key ecosystem defenses such as blacklist-based mechanism~\cite{oest2019phishfarm}.
With so many filtering conditions in the cloaking techniques,
it is very difficult for the content-based anti-phishing systems to acquire real phishing content~\cite{oest2018inside, oest2020phishtime}.
This analysis magnifies an issue that current anti-phishing systems cannot provide a reliable protection for users against phishing websites with cloaking techniques.
So how about we consider the problem from a different angle:
the ultimate goal to mitigate or defeat phishing is to make potential victims not see any phishing content.
So why do we devote ourselves trying to sneak through all the challenges cloaking techniques set and reach to the phishing content?
Why not trigger the cloaking in user's browsers and let phishing websites return a benign-looking web page?
In this way, users will not see phishing content in real time, and a lose-lose situation is shown to phishers because they cannot differentiate visits from between users and anti-phishing crawlers.

To this end, we propose \spartacus, a framework that disguise Internet users as anti-phishing crawlers to request web page content of a suspicious URL, while remain users' own profiles on visiting benign websites.
Before visiting a URL, \spartacus queries the domain information, such as reputation, to decide whether to mutate the HTTP profile.
If not, \spartacus allows the request sent with user's default profile.
Otherwise, \spartacus mutates the items in the HTTP profile that will be inspected by cloaking techniques in the phishing server before sending requests to suspicious URLs.
When the cloaking script examines the HTTP request, it will identify that the visit is from an anti-phishing infrastructure, and will return a benign-looking web page content to users.


\section{Background}

\cloakingTypes



Over past several decades, the anti-phishing ecosystem has proposed and leveraged a number of techniques to detect and migitate phishing attacks~\cite{oest2018inside}.
Techniques such as URL~\cite{bin2010dns, blum2010lexical, huang2012svm, khonji2011novel} and web page content analysis~\cite{wu2006web, zhang2007cantina, zhang2011textual, bilge2011exposure, canali2013role} have raised the defensive level of the ecosystem and productized implementations such as URL blacklists, malicious infrastructure analysis, and e-mail spam filter.

Commodity URL blacklists such as Google Safe Browsing~\cite{whittaker2010large} and Microsoft SmartScreen~\cite{smartscreen} reinforces the backend of the anti-phishing ecosystem, which warn users with prominent sign indicating that the websites ahead are suspicious when phishing URL is detected and blacklisted.
Evasion techniques, also known as cloaking techniques, are widely leveraged in phishing attacks to delay or disable detection by anti-phishing systems~\cite{liang2016cracking, oest2019phishfarm, oest20phishtime}.

\subsection{Server-side Cloaking in Phishing}


Attackers often implement \emph{cloaking techniques} to evade detection by the anti-phishing ecosystem.
Typically, phishing websites with evasion display benign web page content if they suspect the visit originates from a security infrastructure~\cite{wu2005cloaking}.
There are two categories of cloaking techniques, namely server-side and client-side (\autoref{tab:cloakingtypes} shows categories of each type).
Client-side evasion techniques distinguish visitors to display different web page content by executing JavaScript snippets in user's browser~\cite{zhang2021crawlphish}.
Server-side cloaking techniques, on the other hand, identify visits from anti-phishing systems through information in HTTP requests~\cite{oest2018inside, invernizzi2016cloak}.
Typically, server-side cloaking is also known as fingerprinting cloaking, including network, browser, and context~\cite{invernizzi2016cloak}.
\autoref{fig:fp_cloaking} shows how fingerprinting cloaking techniques are used in phishing websites.
Cloaking code embedded in the phishing server fingerprints the profile in the HTTP request and responds different web page content based on the identification of visitors (as either potential victims or anti-phishing crawlers).

Researchers detect and categorize client-side cloaking techniques in phishing by force-executing the payload JavaScript snippets~\cite{zhang2021crawlphish}.
To bypass server-side cloaking techniques in phishing attacks, the anti-phishing systems camouflage themselves as if they are regular visitors.
Only when anti-phishing systems retrieve phishing content, can they classify phishing websites~\cite{xiang2011cantina+,whittaker2010large,smartscreen}.

However, as the development of phishing kits, the size of blocklist in the phishing kit increases.
As shown in~\autoref{fig:servercloaking}, any match of the IP addresses or hostnames in the blocklist will result in an error web page, such as 404 Page Not Found.
Hence, anti-phishing systems can trigger fingerprinting cloaking techniques in the phishing server easily.
As a result, they cannot properly retrieve phishing web page content, which leads to a mis-classification.
Moreover, according to the analysis from Oest et al.~\cite{oest2020sunrise}, it takes anti-phishing crawlers an average of six hours to mark a phishing website as malicious.
The duration is even longer till the last victim visit (18 hours).
Such long reaction time has left phishers a golden hour to hurt Internet users.

We can consider the problem from a different angle: phishers try their best to evade visits from anti-phishing ecosystem, so how about Internet users camouflage themselves as crawlers when visiting phishing websites?
In this way, users can only see an error web page and never submit credentials to phishers.
It also shrinks the reaction time to blacklist them.


% \subsection{Mitigations Against Advanced Phishing}

% \subsection{Limitations of Mitigations}

\begin{figure}
\centering
\includegraphics[width=.9\linewidth]{figs/fp_cloaking.pdf}
\caption{Typical operation of fingerprinting cloaking in phishing websites.}
\label{fig:fp_cloaking}
% \vspace{-15pt}
\end{figure}
\section{Design}


We aim to evade cloaked phishing websites from the user end in real time when users visit them by triggering the fingerprinting cloaking techniques.
To this end, we design, implement, and evaluate~\spartacus, a framework that automatically cloaks users and hence evades malicious websites that implement fingerprint cloaking.

\begin{figure*}
\centering
\includegraphics[width=\linewidth]{figs/arch.pdf}
\caption{\spartacus architecture and its workflow.}
\label{fig:sp_arch}
% \vspace{-10pt}
\end{figure*}

\subsection{Overview}
\autoref{fig:sp_arch} demonstrates the architecture of the \spartacus framework.
The whole framework consists of two parts, \emph{front end} and \emph{back end}. 
The front end is responsible for the functionality performance, such as deciding whether to mutate and mutating profile items in the HTTP request such as User Agent, Referrer, and IP.
% \spartacus also checks the information of benign websites to decide whether to mutate the profile.
In the back end, we maintain three databases to provide support to the front end.
It includes (1) Domain Information Database, which is used to process trustworthy URLs;
(2) Fingerprinting Database, which stores the mutation history;
and (3) Anti-phishing Bot Profile Database, which contains the patterns that advanced phishing websites identify as crawlers.
% phishing attributes, which is used to distinguish the maliciousness of returned web page content. 

When a user tries to visit a URL, at first, the URL goes to blacklist maintained by current anti-phishing systems and \spartacus determines its trustworthiness.
% queries its domain information.
If the existing record is found in the blacklist, the framework will block the access. 
If \spartacus decides the URL to be trustworthy based on the domain information, it will keep user's original profile when requesting the server. 
Otherwise, the front end will query the Fingerprinting Database to see if we have processed the same url before, if found and there exists successful mutation, \spartacus will use recorded profile to request the website, if not, it will mutate profile based on Anti-phishing Bot Profile Database, avoiding failed ones. 
Then with the \spartacus'ed profile, HTTP request is sent to the server. After getting the page content, we determine whether there is maliciousness externally.
The classification methodology stands outside of \spartacus and runs in the background without delaying the web page rendering.
The purpose is to verify the effectiveness of the mutated profile.
% If yes, the browser will show a warning sign to the user, or the web page content will be shown to the user. 
The Fingerprinting Database will be updated accordingly with the used profile and classification result.
% under both situations. 


\subsection{Benign Domain Handling}
\label{ss:benignalg}


We preliminarily tested how \spartacus works on benign websites to figure out the optimal methodology that can bring negligible impact on them.
% Among 60,848 benign websites that \spartacus visited, 150 of them blocked the access from a browser with \spartacus, as false-positives.
% \spartacus also visited 150 cloaked phishing websites, as true-positives.
% To make benign websites allow the visit with \spartacus while keeping its evasion ability against advanced phishing websites,
% To further differentiate between a legitimate anti-bot page and a phishing one that appears due to cloaking,
To further differentiate benign and phishing websites,
we extract domain reputation~\cite{reputation}, domain age~\cite{whois}, and top viewed sub-domains~\cite{topviewedsubdomains} of URLs from benign and malicious websites.
Cisco defines reputation of a domain as five categories: Trusted, Favorable, Neutral, Questionable, and Untrusted.
We randomly select and inspect 150 URLs on each side.
Reputation-wise, 136 out of 150 phishing URLs have reputation lower or equal to Neutral level, the lowest of which is Untrusted Verdict.
In contrast, only 24 of 150 benign URLs reside lower than Favorable level, the lowest of which is Questionable (only one).
In terms of domain lifespan, the mean value of domain duration since registration for benign URLs is 4,692 days and the median is 4,521 days.
However, the average lifespan of the malicious domains is 1,618 days, with median of 900 days.
Moreover, all 150 benign URLs fall into the top viewed sub-domains of the corresponding domain names, but none of the phishing ones matches with them.

From the analysis result, we summarize that a legitimate domain has a higher reputation and longer life than a malicious one, and they are within top viewed sub-domains.
With the finding, we can further reduce the possibility of falsely evading benign websites for \spartacus by querying the attributes of the domain to decide whether to mutate the HTTP profile.
We choose the phishing domain age that resides on 75\% in the list (1,501) and Neutral level as threshold because such threshold can clearly divide trustworthy domains and un-trustworthy ones.
If one URL has a lifespan lower than 1,501 days and its reputation level is Neutral or worse, or its sub-domain is not top viewed, \spartacus will mutate the HTTP profile before request.
The logic is shown in~\Cref{alg:mutatelogic}.

\begin{algorithm}\captionsetup{labelfont={sc,bf}, labelsep=newline}
  \caption{Logic of mutating HTTP profile}
\begin{algorithmic}[1]

\State $p = default\_profile$
\State $u = url\_to\_visit$

\If{\State ($reputation(u)~\leq Neutral$) \texttt{and}
\State ($duration(u)~\geq 1,501$) \texttt{or} 
\State ($d.domain$ NOT in top reviewed sub-domains)}
    \State $p = mutate\_http\_profile(p)$
\EndIf

\State $send\_request(p)$

\end{algorithmic}
\label{alg:mutatelogic}
\end{algorithm}



\subsection{Back End Databases}

In the back end of \spartacus framework, we maintain three databases to facilitate the functionalities performed in the front end.

\noindent
\textbf{Domain Information Database.}
As discussed in~\Cref{ss:benignalg}, to minimize the potential impact on benign website visit,
\spartacus needs to query the reputation, domain age, and popular subdomains for each domain the user tries to visit.
To further reduce the latency where \spartacus has to send requests to the reputation lookup database externally and waits for the response,
we beforehand cache the information of top 1,000 domains in Alexa Top One Million List~\cite{AlexaTop1M} into \spartacus's backend.
The record in the database is formed as (\emph{domain}) - (\emph{reputation}) - (\emph{age}) - (\emph{top reviewed sub-domains}).
In this way, the information of top visited websites \spartacus needs can be acquired quickly.


\begin{comment}

\noindent
\textbf{Blacklist database.}
Blacklist based anti-phishing systems such as Google Safe Browsing and Microsoft SmartScreen have been examining and collecting phishing URLs to warn users before they visit the websites.
We collect known phishing URLs from common blacklists so that \spartacus can leverage the achievements from the anti-phishing systems to facilitate the web request process when dealing with known phishing websites.
% Similarly, we leverage a whitelist to avoid any impacts brought by \spartacus so that the users can visit legitimate websites with their own profiles.

\end{comment}


\noindent
\textbf{Fingerprinting Database.}
The Fingerprinting Database is shared among users who leverage \spartacus.
It stores information that users can or cannot see malicious content when requesting the server.
Every record in the database is formed as \emph{(hashed URL)} - \emph{(used bot profile)} - \emph{(success nor not)}.
After the maliciousness classification, if phishing content is not found in the responded content, then we mark the mutated profile used in the request this time, connect it with the URL, and update the pair as success into the Fingerprinting Database.
When other users try to visit the \emph{same} URL, \spartacus will use the corresponding profile recorded in the Fingerprinting Database to prevent users from receiving malicious content.

\noindent
\textbf{Anti-phishing Bot Profile Database.}
The profiles of anti-phishing system crawlers keep updating all the time.
Accordingly, phishers refresh their blacklist to precisely evade the access from the anti-phishing bots.
Therefore, we crawl the up-to-date patterns of anti-phishing systems~\cite{crawlerinfo} and blocked word list from retrieved phishing kits into the database.
We also include the IP of proxy servers \spartacus uses in the database.
And hence \spartacus can leverage newest patterns to visit suspicious websites and receive benign web page content by triggering the cloaking techniques.

% \noindent
% \textbf{Phishing Attribute Database.}
% After sending request with mutated profile and receiving respond from the server, \spartacus needs to classify whether maliciousness exists in the web page content.
% It implements the state-of-the-art phishing detections such as content-based and visual-similarity-based methodologies to extract phishing attributes based on the responded web page content such as screenshots and HTML source code.
% \spartacus compares such features with those from websites of most phishing-targeted organizations, which stores in the Phishing Attribute Datatbase.





\subsection{Decision Makers on A URL Visit}

When a user attempts to visit a URL, the URL will go through several decision makers to assure a safe browsing.
The decision makers consist of Blacklist \& Trustworthy Filter and Database Query.
% , and Maliciousness Classification.
We will elaborate them in order of the system flow.

\noindent
\textbf{Blacklist \& Trustworthiness Filter.}
With the contribution of the anti-phishing ecosystem, we can filter known phishing URLs that have already been blacklisted by commodity blacklists such as Google Safe Browsing and Microsoft SmartScreen.
% through the query to the Malicious URL BlackList Database.
Any match in the blacklist database will result in a block access to the URL without further action.
Additionally, as discussed in~\Cref{ss:benignalg}, to minimize the impact on benign domains, \spartacus does not mutate the profile if it decides that the URL is benign based on the domain information.

% \noindent
% \textbf{Whitelist Filter.}
% Similar to the blacklist filter, the whitelist filter in \spartacus helps filter the whitelisted URLs so that the user can receive the web page content with best out-looking that the organization configures to fit in the visit from different browsers or devices.

\noindent
\textbf{Database Query.}
When the URL does not fall into either Blacklist or Trustworthiness filter, \spartacus will examine whether such URL has been visited once by either the user or other users by querying the Fingerprinting Database.
If a record is found in the Database, \spartacus will mutate the web request profile according to the record to make sure that such web request profile will return a benign web page content;
otherwise, \spartacus will call a \emph{Bot Profile Mutator}, which selects one profile from Anti-phishing Bot Profile Database to submit the web request and lowers the impact to possible web page layout to the lowest.



\subsection{Bot Profile Mutator}

It is the Bot Profile Mutator that takes responsibility of profile mutation to evade malicious content in advanced phishing websites.
There are several items in the HTTP profile that can be modified or changed to camouflage the user as anti-phishing crawlers.
Take User-Agent as an example. 
Generally, anti-phishing crawlers contain ``bot'', ``crawler'', or the name of the company such as ``Google'', ``Facebook'' in the User-Agent string.
We name the words that can trigger the cloaking technique in phishing attacks as \emph{triggering words} (e.g., words shown in~\autoref{tab:topsenswords}).
Scripts in the phishing server typically read the string from HTTP requests and filter visits that contain crawler-like patterns.
Another instance is Referrer, where the visit is from previously.
Typically the previous location for a potential victim is from phishing lures the attackers distributed earlier.
Hence, phishers can block all visits who are not from the phishing lures.
At last, the profile mutator can leverage proxy servers to camouflage user's HTTP request.
For example, a proxy server established in \emph{Amazon AWS} is very useful because phishers have knowledge that some anti-phishing crawlers are built in AWS EC2 (according to the finding from~\autoref{tab:topsenswords}).
In this case, the mutator can show the IP where crawlers are often from to the phishing server to trigger the cloaking techniques.

To take advantage of phisher's filtering algorithm, the Bot Profile Mutator modifies the identifiable items in the HTTP profile such as User-Agent and Referer before the browser sends out the request.
In detail, after querying from the Fingerprinting Database, \spartacus will append the successful triggering word in the Database to user's own User-Agent string.
It also checks the necessity of mutation of Referer within the same record.
If there is no successful record in the database,
the mutator will choose one word from a list of 407 triggering words, following the order of popularity in~\autoref{tab:topsenswords}.
Additionally, \spartacus sets the Referer to either None or ``www.google.com'' to further pretend the user as anti-phishing crawler.
% 407 sensitive words
As for IP mutation, \spartacus can reroute the request to a proxy server, whose IP is one of the most popular IPs in the blocklist of phishing kits.

% \subsubsection{Maliciousness Classification.}
% \noindent
\subsection{Maliciousness Classification}
After submitting the web request to the server, \spartacus along with the browser will receive a response from the server.
There are different situations of the response as follows:

\begin{itemize}
    \item The server responds a benign-looking content, such as an error page, 
    % or an error page with status code of 4XX/5XX, 
    or it redirects the visit to a benign website, which is usually the website it impersonates;
    \item The server returns malicious web page content.
\end{itemize}


The former result is because (a) the server is malicious with evasion techniques and determines the visit from \spartacus as anti-phishing bot visit, or (b) the website is benign itself.
The latter one is caused by (a) a malicious website without evasion techniques, or (b) a malicious website with evasion techniques, which is not triggered by the mutated profile from \spartacus.

The latter situation is what \spartacus needs to handle.
As for reason (a), \spartacus relies on the current anti-phishing systems that are well capable of detecting basic phishing websites within a short time period~\cite{oest2020phishtime} and hence the whole ecosystem can be protected from such phishing attacks.
For reason (b), the phishing website implements evasion techniques that remain unknown after one visit.
However, as \spartacus can be used by a large number of users, there is high possibility where a number of users will visit the same URL.
\spartacus learns from different attempts of mutations of web request profiles that failed to trigger cloaking in the Fingerprinting Database.
Avoiding using the failed ones, \spartacus mutates the profile with new triggering words from the list, and hence we can eventually fingerprint the trigger of the evasion technique in the phishing website and apply such configuration for all \spartacus users who visit the same URL.
% Before \spartacus figures out the cloaking technique, it implements visual-similarity and content-based phishing detection mechanisms and alerts the user if any of the attributes detected in the content by matching the features with those in Phishing Attribute Database.
% otherwise, the browser user will see a benign web page content.






\section{Privacy}

In this section, we discuss the sensitive information \spartacus requires from users to support the phishing evasion services and how it protects the privacy.
% Note that installing and using \spartacus may lead to the compromise of privacy.
% For example, some discount discovery extension may leak the annual income level of users~\cite{honey}.
% And some of the VPN services require user's location information and Personal Identifiable Information (PII)~\cite{ZenMate}.
% Such privacy information may be leaked and hence abused by attackers.  
% We thereby enumerate the privacy information \spartacus collects to support the phishing evading services.

\subsection{Privacy Information}

\spartacus requires four types of sensitive information from users to prevent them from being trapped in advanced phishing websites.
We do not collect any PII in \spartacus and hence can protect user's privacy to maximum extent.

% \noindent
(1)~\emph{URL a user tries to visit}: \spartacus needs it to query the reputation, age of domain, and top reviewed subdomains to determine whether to mutate.
If \spartacus makes the decision to mutate the profile, it also needs the \emph{hashed} URL to query the Fingerprinting Database for successful mutation that can evade malicious content.

% \noindent
(2)~\emph{HTTP profile used in the request}:
After \spartacus decides to mutate the user's HTTP profile before visiting a suspicious website, it requires the user's profile to modify it.
The privacy information in the profile includes the User-Agent string, where browser version, browser type, and operating system information reside, and the Referer, which implies where the user comes from.
By accessing the profile, \spartacus can modify the fingerprints that phishers identify as anti-phishing crawlers to camouflage users.

% \noindent
(3)~\emph{Returned HTTP response}:
\spartacus requires the returned HTTP response to inspect whether the website still contains maliciousness.
% For example, if the response status code is 4XX/5XX, it means that there will not be malicious content.
If there is valid content returned,
an external classification process will determine its maliciousness by searching for phishing content such as sensitive words and log in form~\cite{xiang2011cantina+}.
The inspection result marks the corresponding profile mutation successful or not.

(4)~\emph{URL and profile share}:
\spartacus at last uploads the \emph{hashed} URL and the inspection result along with the corresponding HTTP profile to the server.
In the database, we only store the triggering word \spartacus uses, whether to blank Referrer, and which proxy server to reroute.
We do not save any user-related information in the Fingerprinting Database.
When other users visit the same URL, the successful mutation will be shared to them.
If there is no successful variant, \spartacus will avoid to use the unsuccessful ones to mutate.


\subsection{Privacy Consent and Protection}

We ensure that our \spartacus system well considers users' privacy information,
so we have both consent and protection methodology to notice users and prevent their information from being stolen and abused.

\noindent
\textbf{Consent.}
To have users be aware of the types of privacy information \spartacus checks and modifies before using it,
we set up a privacy policy consent notice when people first time use \spartacus.
At the beginning of the consent, we summarize the privacy information \spartacus use for user's convenience.
Then, by using Privacy Policies~\cite{privacypolicy}, we create one privacy policy for \spartacus.
We elaborate the information \spartacus collects, how it will be used, and how it will be transferred and shared in the privacy police page.
Users can choose to opt out and uninstall our system if they do not agree with the consent.

\noindent
\textbf{Protection Methodology.}
First of all, the privacy information \spartacus collects does not contain any PII, which minimizes potential harm.
Secondly, when the URL and its corresponding HTTP profile is transferred to the server,
it is hashed first and uploaded to the server.
Lastly, when other users look up existing records from the server,
their URL is also hashed first and looked up in the database in hash version.
Furthermore, what they receive from the Fingerprinting Database is whether specific sensitive words can evade maliciousness.
They will not know who exactly ever visited the website.
And hence, the privacy of the user can be protected.
\section{Evaluation}

We implement \spartacus in terms of Chrome extension, which the anti-phishing ecosystem can leverage and spread trivially.
and evaluate it from three perspectives, including effectiveness, user experience, and privacy.
These three aspects demonstrate the feasibility of the \spartacus system because it can successfully evade advanced phishing websites, can negligibly introduce latencies to the users when visiting websites, and can protect privacy of the users.

\subsection{Prevalence of Fingerprinting Cloaking}

\begin{figure}
\centering
\includegraphics[width=\linewidth]{figs/server_cloaking.png}
\caption{PHP code snippet of fingerprinting cloaking in a phishing kit.}
\label{fig:servercloaking}
% \vspace{-10pt}
\end{figure}

To understand the prevalence of cloaking techniques used in advanced phishing kits, we manually inspect 56 phishing kits, and extract common patterns that implement fingerprinting cloaking techniques in phishing kits.

The patterns include (1) sensitive file names such as ``antibot'', ``blocker'', and ``antirobot''; (2) blocked words, for example, an array containing crawler information such as ``google'', ``paypal'', or ``phishtank'' (e.g., \texttt{blocked\_words} in~\autoref{fig:servercloaking}); (3) IP checker that blocks visit from certain IP addresses, such as \texttt{bannedIP} in~\autoref{fig:servercloaking}; and (4) error header, returning an error status code and an error web page, such as \texttt{header} function in PHP.
These attributes reflect the implementation of fingerprinting cloaking techniques in phishing kits.

With these patterns, we can automatically inspect such evasion to evaluate its prevalence.
Among the inspected kits in~\emph{phishunt.io}~\cite{phishunt} from January to June 2021, 198 of 236 contain cloaking techniques to fingerprint anti-phishing crawler visit through IP, Referrer, and/or User-Agent.
To comprehensively measure the prevalence of server-side fingerprinting cloaking techniques in phishing kits,
we also run our pattern finding script on Cisco's dataset that is contributed to Virus Total.
Preliminarily, 100 out of 100 phishing kits from Cisco's dataset contain such evasion.
% , while only one implements server-side click-through cloaking.

From this experiment,
we show that phishers prefer to fingerprint visit through information in HTTP request to distinguish humans out of crawlers.
And thanks to the fact, \spartacus can evade phishing websites generated from such advanced phishing kits.


% \subsection{Support From the Ecosystem}

% \spartacus focuses on the evasion of advanced phishing attacks,
% because the current anti-phishing techniques can effectively and efficiently detect and blacklist basic phishing~\cite{oest2020phishtime}.

% We still need to measure the ability of current anti-phishing systems against basic attacks.
% To this end, we submit phishing URLs to one of the systems, Google Safe Browsing (GSB), the same time when \spartacus inspects them.
% Meanwhile, we keep querying GSB for the blacklist result to calculate the blacklist speed.



\subsection{Effectiveness}


\begin{figure*}
\centering
	\begin{subfigure}[tb]{.31\textwidth}
		\includegraphics[width=\linewidth]{figs/netflix_n.pdf}
        \caption{Default browser visit.}
        \label{fig:normal}
	\end{subfigure}%
	\quad
	\begin{subfigure}[tb]{.31\textwidth}
		\includegraphics[width=\linewidth]{figs/netflix_sp.pdf}
        \caption{\spartacus'ed browser visit with error.}
        \label{fig:sp1}
	\end{subfigure}%
	\quad
	\begin{subfigure}[tb]{.31\textwidth}
		\includegraphics[width=\linewidth]{figs/netflix_sp2.png}
        \caption{\spartacus'ed browser visit with content.}
        \label{fig:sp2}
	\end{subfigure}%
	\quad
	\vspace{-5pt}
	\caption{Web page contents from default and \spartacus'ed browser visits for a cloaked phishing website.}
	\label{fig:effectiveness}
	\vspace{-10pt}
\end{figure*}

We evaluate the effectiveness of \spartacus by conducting an experiment where we visit same phishing websites with different configurations of automated browsers, one with default setting, the other with \spartacus to mutate the profile.
We simulate the default setting automated browser as normal users and the one with \spartacus as users with our system.
The phishing URLs both browsers visit are from APWG, which is an open sourced dataset for reported phishing URLs.
For each visit, we record the final landing web page content and URL.
Either a legitimate final URL or a non-malicious web page content indicates the success of \spartacus evading phishing content for users.
If there is still malicious content shown to \spartacus, it indicates that either the phishing website does not contain advanced evasion techniques, or \spartacus does not trigger them with current fingerprint.
For the first situation, the current anti-phishing ecosystem has proposed methodologies to mitigate basic phishing attacks with effectiveness and efficiency.
As for the second, we believe that by visiting the same website with different configurations of HTTP requests mutated by \spartacus, the evasion techniques embedded in the phishing website can be triggered eventually. 

We examined 160,728 phishing URLs from APWG from November 2020 to July 2021,
132,247 of which do not contain malicious content.
We consider an HTTP response benign if (1) its web page content does not contain sensitive words or bad forms, according to CANTINA+~\cite{xiang2011cantina+};
(2) the response status code is an error one (4XX/5XX);
or (3) the destination domain is legitimate, excluding web hosting service domains.
\autoref{fig:effectiveness} demonstrates the difference of response web page content between default and \spartacus'ed browser visit for a cloaked phishing websites.
The content in~\autoref{fig:normal} shows the phishing content when a real person visits it.
On the other hand, when \spartacus appends a bot-looking string to the existing User-Agent, such as~\emph{googlebot}, the phishing server considers the visit as from an anti-phishing infrastructure.
Thus, it denies the request from \spartacus and an error web page is shown as~\autoref{fig:sp1}.
On the other hand, advanced phishing servers can redirect visit to a benign domain with status code of 200 instead of returning an error.
In this way, \spartacus'ed browser will receive web page contents such as~\autoref{fig:sp2},
which also indicates a successful evasion 
from \spartacus.

It shows that \spartacus can impersonate anti-phishing crawler and trigger the cloaking techniques of advanced phishing websites.
Because the web page content returned through \spartacus's visit does not contain any maliciousness, Internet users will not be trapped in the attack.

% \textcolor{blue}{eval result}

\subsection{Support from Ecosystem}

Note that there are 17.72\% URLs that \spartacus cannot evade.
We hypothesize that these websites are basic phishing that do not contain cloaking techniques.
The anti-phishing systems nowadays such as Google Safe Browsing and VirusTotal can timely detect and/or blacklist traditional phishing attacks.
Therefore, they can handle the phishing websites that cannot be evaded by \spartacus.
And hence, \spartacus with the support of the anti-phishing ecosystem can prevent all types of phishing.

To verify our hypothesis, we evaluate the blacklist speed of current anti-phishing systems on the examined phishing URLs.
Due to the deployment of this experiment, we did not submit all phishing URLs.
Among 4,674 submitted phishing URLs that are not evaded by \spartacus, anti-phishing systems such as Google Safe Browsing and VirusTotal can blacklist 4,598 of them.
% The median detection speed is 28 minutes.
The rest 76, after our manual inspection, is found that they are falsely reported to APWG.
The evaluation result verifies that the ecosystem currently can protect users from being trapped in basic phishing attacks.
In contrast, we totally submitted 40,852 URLs that can be evaded by \spartacus.
The result shows that 24,154 of them are not detected or blacklisted by the anti-phishing systems.
% For URLs that can be detected, the median speed is 154 minutes, compared with that of 22 minutes for basic phishing websites.



\autoref{fig:venn_support} overviews the ability of \spartacus and support from the anti-phishing ecosystem.
% Among all 45,526 phishing URLs submitted to anti-phishing systems,
% 21,296 of them can be detected or blacklisted.
% Due to the imbalance of the amount between evaded and not evaded phishing URLs,
% we scale the not evaded ones and conduct the analysis.
% Although the ecosystem can detect or blacklist 21,296 (46.77\%) of the phishing URLs, 
Within our dataset, advanced phishing websites have overwhelmed the basic ones, which is shown in~\autoref{fig:venn_support} as blue circle and red circle, respectively.
However, the anti-phishing ecosystem can only detect 40.87\% of the submitted evaded phishing URLs.
As a comparison, \spartacus can evade 89.73\% of submitted phishing URLs.
% But the ecosystem can only detect or blacklist 46.77\% of them.
For the rest that \spartacus cannot evade,
the ecosystem can mostly handle the not-evaded phishing URLs as a complement of \spartacus.
With \spartacus and the support from current anti-phishing systems, 
both advanced and basic phishing websites can be evaded or detected.

We visualize the detection speed of current anti-phishing systems in~\autoref{fig:evade_bl_time}.
All submitted phishing websites that \spartacus cannot evade can be detected/blacklisted in two hours.
50\% of these websites can be detected within 22 minutes.
As a comparison, current anti-phishing systems do not perform well against phishing websites that can be evaded by \spartacus.
The median detection time is 154 minutes, but it can take as long as 47.82 hours to detect the other half.
It reflects the ability of the anti-phishing ecosystem against phishing websites: 
for basic phishing websites, they can react and blacklist them timely;
for advanced phishing, it takes a long time to detect, which can be exploited by phishers to lure victims.
Therefore, \spartacus not only can protect users from advanced phishing websites during the golden hour~\cite{oest2020sunrise} left by the current anti-phishing ecosystem due to process time, but also can evade phishing content even if the ecosystem cannot detect.


\begin{figure}
\centering
\includegraphics[width=\linewidth]{figs/venn_eval_3.pdf}
\caption{Venn Diagram describing the evasion from \spartacus and support from the anti-phishing ecosystem.}
\label{fig:venn_support}
% \vspace{-10pt}
\end{figure}

\begin{figure}
\centering
\includegraphics[width=\linewidth]{figs/evade_bl_time_total.pdf}
\caption{CDF of Blacklist/Detection time of current anti-phishing systems against phishing URLs evaded by \spartacus}
\label{fig:evade_bl_time}
% \vspace{-10pt}
\end{figure}


\subsection{Efficiency and Latency}

We need to make sure that \spartacus will not impact negatively on the user experience when they visit benign websites.
From the design of \spartacus, it will introduce latencies, including database query, HTTP profile mutation, and returned content inspection.
However, the latencies \spartacus introduces can be negligible.
Therefore, we conduct an experiment to measure the latency for the \spartacus system for the following three perspective, database query, profile mutation, and content inspection.
We leverage \emph{exthouse}~\cite{exthouse}, which analyzes the impact of a browser extension on web performance, as our test bench.
% We run exthouse with 1,000 phishing websites and 1,000 benign websites on two configurations of browsers, one with \spartacus, the other without, and compare the mean rendering time for each group of websites.
It contains three major measurements:
(1) Time to Interactive (TTI): the time it takes for the page to become fully interactive with the extension; 
(2) First Input Delay (FID $\Delta$): the time from when a user first interacts with the website to the time when the browser is actually able to begin processing event handlers in response to that interaction;
and (3) Scripting Time (Scripting $\Delta$): the amount of time JavaScript execution in the extension.
The lower these three factors are, the better the website performs with the tested extension.
At last, \emph{exthouse} will score of the extension.
A higher score reflects a better performance of the extension.

\exthouse

\autoref{tab:exthouse} illustrates the metrics of top 10 Chrome extensions~\cite{exthouse} along with \spartacus under the inspection of \emph{exthouse}.
We test these extensions with 100 websites, including both benign and malicious, and average the metrics.
\spartacus has a score of 100, 0 FID and scripting delta, and 800 ms of TTI.
It means that with \spartacus, users can still interact with the website with minimum latency.
The inspection result shows that \spartacus out-scores popular Chrome extensions and does not impact the performance of websites, compared with other extensions.

\subsection{Impact on Benign Website}

Besides evading malicious content in phishing websites, \spartacus is also required to minimize the negative impacts on benign URL visit.
They can include ability of access the website, correct display of website layout, and correct functionalities of the website.
To evaluate the potential impacts to benign website, we conduct two experiments: Coarse-Grained and Fine-Grained.

\subsubsection{Coarse-Grained Experiment}

\coarsegrain

In Coarse-Grained experiment, we intend to evaluate if \spartacus system has negative impact on the access to the website or the website layout.
So we visit 60,848 (9.66\%) out of 629,843 URLs in Alexa Top One Million Domain List~\cite{AlexaTop1M} in both default and \spartacus'ed browsers.
We compare the web page screenshot and HTML similarity on the visited URLs.
The result is shown in~\autoref{tab:coarsegrain}.
Among them, 0.25\% (150) block the access from \spartacus'ed browser;
0.20\% (124) have different layouts.

\begin{figure}[t]
    \centering
    \begin{subfigure}{0.48\linewidth}
        \centering
            \includegraphics[width=\textwidth]{figs/38_normal.png}%
        \caption{Default browser visit.}
        \label{fig:coarse_normal}
    \end{subfigure}
    ~
    \begin{subfigure}{0.48\linewidth}
        \centering
            \includegraphics[width=\textwidth]{figs/38_sp.png}
        \caption{\spartacus visit}
        \label{fig:coarse_sp}
    \end{subfigure}
    
    \caption{Difference due to the shape of buttons.}
    \label{fig:coarse}
    % \vspace{-10pt}
\end{figure}

\begin{figure}[t]
    \centering
    \begin{subfigure}{0.48\linewidth}
        \centering
            \includegraphics[width=\textwidth]{figs/2306_normal.png}%
        \caption{Default browser visit.}
        \label{fig:2306_normal}
    \end{subfigure}
    ~
    \begin{subfigure}{0.48\linewidth}
        \centering
            \includegraphics[width=\textwidth]{figs/2306_sp.png}
        \caption{\spartacus visit}
        \label{fig:2306_sp}
    \end{subfigure}
    
    \caption{Difference due to popup.}
    \label{fig:2306}
    % \vspace{-10pt}
\end{figure}

We first examine the reason why \spartacus'ed browser show different layouts from the default browser.
And we find that although the screenshot and HTML similarity is not high between default and \spartacus browser visits,
to people, such difference does not impact them using or browsing the website.
For example, two web pages are different in terms of screenshot similarity between default and \spartacus browser visits shown in~\autoref{fig:coarse}.
The difference is due to the shape of buttons and different of background color.
% The change is on the scrolling promotion ads in the \texttt{div} tag.
% When the screenshot was captured in default visit, Ad One happened to show;
% and another Ad was in the way when \spartacus took the screenshot.
Similarly, in~\autoref{fig:2306}, a window popped up to ask for the permission of cookie in default browser, but did not in \spartacus's visit.
The cookie request pop-up is missing in \spartacus browser, not due to the extension itself, but because we visit the same website 10 times in different browsers without \spartacus, only for 3 times the pop-up appeared.
Even though our evaluation script can distinguish the difference between visits from two browsers,
users will not perceive that.
They can normally interact with the websites with \spartacus providing phishing-evading services.
% , which indicates that \spartacus does not make users perceive the difference, even if it has 


To further differentiate between a legitimate anti-bot page and a phishing one that appears due to cloaking,
we extract domain reputation~\cite{reputation}, registration duration~\cite{whois}, and top viewed sub-domains~\cite{topviewedsubdomains} of URLs from false-positive (FP) and true-positive (TP) sides.
Cisco defines reputation of a domain as five categories: Trusted, Favorable, Neutral, Questionable, and Untrusted.
We inspect 150 URLs on each side.
Reputation-wise, 136 out of 150 TP URLs --- phishing --- have reputation lower or equal to Neutral level, the lowest of which is Poor Verdict.
In contrast, 24 of 150 FP URLs reside lower than Favorable level, the lowest of which is Questionable (only one).
In terms of domain lifespan, the mean value of domain duration since registration for FP URLs is 4,692 days and the median is 4,521 days.
However, the average lifespan of the malicious domains is 1,618 days, with median of 900 days.
Moreover, all 150 FP URLs fall into the top viewed sub-domains of the corresponding domain names, but none of the TP ones matches with them.

From the evaluation result, we summarize that a legitimate domain has a higher reputation and longer life than a malicious one, and they are within top viewed sub-domains.
With the finding, we can further reduce the FPs of \spartacus by querying the attributes of the domain to decide whether to mutate the HTTP profile.
We choose the phishing domain duration that resides on 75\% in the list (1,501) and Neutral level as threshold.
If one URL has a lifespan lower than 1,501 days and its reputation level is Neutral or worse, or its sub-domain is not top viewed, \spartacus will mutate the HTTP profile before request.
The logic is shown in~\Cref{alg:mutatelogic}.

\begin{algorithm}\captionsetup{labelfont={sc,bf}, labelsep=newline}
  \caption{Logic of mutating HTTP profile}
\begin{algorithmic}[1]

\State $p_d = default\_profile$
\State $u = url\_to\_visit$

\If{\State $reputation(u)~\leq Neutral$ \texttt{and}
\State $duration(u)~\geq 1,501$ \texttt{or} 
\State $d.domain$ NOT in top reviewed sub-domains}
    \State $p_n = mutate\_http\_profile(p_d)$
\EndIf

\State $send\_request(p_n)$

\end{algorithmic}
\label{alg:mutatelogic}
\end{algorithm}

We re-run the evaluation based on the threshold and find that 121 out of the 150 legitimate domains show same web page content on both default and \spartacus'ed browsers, as listed in~\autoref{tab:coarsegrain}.
Hence, only 0.04\% of 60,848 domains are falsely evaded.
Additionally, 39 of the domains show different layouts between default and \spartacus browsers.
As we discussed above, such difference will not affect the activities of normal users,
but after applying~\Cref{alg:mutatelogic}, \spartacus and normal browsers perform more alike on benign domains.
As comparison, all the phishing URLs can still be evaded through \spartacus because they all triggered the condition to invoke \emph{mutate\_http\_profile()}.

\subsubsection{Fine-Grained Experiment}

\finegrain


In Fine-Grained experiment, we aim to exercise and evaluate the operation of websites visited through \spartacus.
It was inspired by the methodology used by Snyder et al.~\cite{snyder2017most} and Trickel et al.~\cite{trickel2019everyone}.
This methodology concentrates on the operation of a website from the perspective of the user.
Even though \spartacus may introduce an error to a website while users do not perceive any difference when browsing, then we consider that \spartacus does not impact negatively on the website.
% evaluate potential impact \spartacus brings on the functionalities of benign domains.
This method of evaluation focuses more on potential impact \spartacus brings on the functionalities of benign domains than how \spartacus works.
And it was performed by the authors manually.

The experiment includes the evaluation of visibility of websites and interactions between visitor and the website.
It brings an additional metric that evaluates how \spartacus will influence users' daily activities.
There are four steps in the experiment.
(1) We open legitimate domains in a browser with \spartacus installed and also in one with default settings.
(2) We inspect the accessibility of the website, similar to the Coarse-Grained experiment.
(3) After the success of web page content retrieval, we compare the layouts between different visit.
(4) We interact with links, buttons, and other activities such as register/login, online chat, or shopping to make sure that their functionalities perform correctly.
(5) At last, we test the authentication functionality to make sure that \spartacus extension will not impact it.

We randomly selected 60 domains from Alexa Top One Million List, especially 20 every 200 thousands.
The result is displayed in~\autoref{tab:finegrain}.
As a comparison, default browser has the same result as that of \spartacus.
Among the 60 legitimate domains, we can access 58 of them.
Two domains are inaccessible even in the default browser, so we suspect them offline already.
For the accessible 58 domains,
we follow the steps mentioned above to inspect them.
All of them have the same layout as the visit from the default browser.
Then we interact the 58 websites by clicking the buttons, chatting online, and adding items into the cart if it is a shopping website.
All 58 websites performed well.
At last, we register an account on XYZ websites and they all allow us to do so.
Even though we can successfully register an account, we still need to make sure that we can log in properly with those accounts to test the authentication process under \spartacus.
The result shows that all the accounts we registered during the Fine-Grained experiment can be logged in successfully.
It means that the \spartacus system does not impact the authentication process in the website.

In summary, with the results retrieved from both coarse- and fine-grained experiment, we can summarize that \spartacus does not impact the accessibility and visibility on high reputation or long age domains, or its top reviewed subdomains.
Besides, for domains who can be accessed and displayed properly, the functionalities including links, buttons, and registration and authentication in the websites will not be affected.
Therefore, \spartacus can protect users from visiting advanced phishing websites while keep their normal browsing activities.

% \noindent
% \textbf{Database Query}.
% According to the e

% \noindent
% \textbf{Profile Mutation}.

% \noindent
% \textbf{Content Inspection}.

% \subsection{Privacy}


\section{Related Work}

Researchers have been studying phishing for several decades.
They have proposed several methodologies to detect phishing attacks based on one or some of features from URL, content, and etc, and then warn users before visiting the deceptive websites.
Some work analyzes URL of a suspicious website based on the lexical features or URL ranking to determine the maliciousness of the site~\cite{blum2010lexical, le2011phishdef, khonji2012enhancing, feroz2015phishing}.
Others collect web page content and detect phishing websites with textual and visual similarity features~\cite{zhang2007cantina, zhang2011textual, dunlop2010goldphish}.
Combining with available features including both URL and web page content, researchers have developed blacklist based anti-phishing systems such as Google Safe Browsing~\cite{whittaker2010large} to protect Internet users from visiting suspicious websites.
All the proposed methodologies in the past, however, have a limitation that they are detection systems, and require certain features to classify the maliciousness,
which takes long time to output and hence leaves phishers golden hour to harvest.

With the large-scale implementation of cloaking techniques in phishing attacks~\cite{oest2020sunrise, oest2020phishtime, oest2019phishfarm, oest2018inside}, researchers realize that the sophisticated phishing attacks are responsible for substantial portion of damage and that the whole ecosystem should prioritize on mitigating phishing with evasion techniques.
The cloaking techniques make the path of anti-phishing tougher for the proposed methodologies because it becomes more and more difficult to retrieve phishing content, which most anti-phishing systems depend on.
With very limited amount of features of a suspicious website, the system cannot determine precisely the maliciousness.

Therefore, analysis and detection of server-side
% ~\cite{wang2011cloak, invernizzi2016cloak} 
and client-side
% 
cloaking techniques have been proposed to fight against such sophistications.
For client-side cloaking techniques, Zhang et al.~\cite{zhang2021crawlphish} proposed CrawlPhish to force-execute JavaScript snippets in the payload to reveal malicious content.
As for server-side cloaking in phishing, previous work~\cite{wang2011cloak, invernizzi2016cloak, oest2018inside} can only categorize types of server-side cloaking through analysis of compromised phishing kits.
The whole security community does not have an efficient and effective methodology to mitigate them.
Phishers can always evolve to add new crawler-looking features in the phishing kits to block any suspicious visit.


% All the proposed methodologies, however, have a limitation that they only provide a pain-relief pills to certain symptoms of phishing attacks, such as suspicious URL, similar layout as the impersonated website, and cloaking techniques.

% Furthermore, as phishers have been developing advanced cloaking techniques, it becomes more and more difficult to retrieve phishing content, which most anti-phishing systems depend on.

% Therefore, with the contribution of previous studies on phishing symptoms, 
Considering the nature and prevalence of cloaked phishing websites~\cite{oest2018inside, oest2019phishfarm},
our work provides a vaccine to neutralizing advanced phishing-virus.
Rather than trying best to bypass cloaking techniques in phishing websites, \spartacus deliberately triggers them and hence retrieve benign content to show to users.
Our framework is also extendable and always up-to-date by adding fingerprints that researchers will find in the future.



% %-------------------------------------------------------------------------------
% \section{Introduction}
% %-------------------------------------------------------------------------------

% A paragraph of text goes here. Lots of text. Plenty of interesting
% text. Text text text text text text text text text text text text text
% text text text text text text text text text text text text text text
% text text text text text text text text text text text text text text
% text text text text text text text.
% More fascinating text. Features galore, plethora of promises.

% %-------------------------------------------------------------------------------
% \section{Footnotes, Verbatim, and Citations}
% %-------------------------------------------------------------------------------

% Footnotes should be places after punctuation characters, without any
% spaces between said characters and footnotes, like so.%
% \footnote{Remember that USENIX format stopped using endnotes and is
%   now using regular footnotes.} And some embedded literal code may
% look as follows.

% \begin{verbatim}
% int main(int argc, char *argv[]) 
% {
%     return 0;
% }
% \end{verbatim}

% Now we're going to cite somebody. Watch for the cite tag. Here it
% comes. Arpachi-Dusseau and Arpachi-Dusseau co-authored an excellent OS
% book, which is also really funny~\cite{arpachiDusseau18:osbook}, and
% Waldspurger got into the SIGOPS hall-of-fame due to his seminal paper
% about resource management in the ESX hypervisor~\cite{waldspurger02}.

% The tilde character (\~{}) in the tex source means a non-breaking
% space. This way, your reference will always be attached to the word
% that preceded it, instead of going to the next line.

% And the 'cite' package sorts your citations by their numerical order
% of the corresponding references at the end of the paper, ridding you
% from the need to notice that, e.g, ``Waldspurger'' appears after
% ``Arpachi-Dusseau'' when sorting references
% alphabetically~\cite{waldspurger02,arpachiDusseau18:osbook}. 

% It'd be nice and thoughtful of you to include a suitable link in each
% and every bibtex entry that you use in your submission, to allow
% reviewers (and other readers) to easily get to the cited work, as is
% done in all entries found in the References section of this document.

% Now we're going take a look at Section~\ref{sec:figs}, but not before
% observing that refs to sections and citations and such are colored and
% clickable in the PDF because of the packages we've included.

% %-------------------------------------------------------------------------------
% \section{Floating Figures and Lists}
% \label{sec:figs}
% %-------------------------------------------------------------------------------


% %---------------------------
% \begin{figure}
% \begin{center}
% \begin{tikzpicture}
%   \draw[thin,gray!40] (-2,-2) grid (2,2);
%   \draw[<->] (-2,0)--(2,0) node[right]{$x$};
%   \draw[<->] (0,-2)--(0,2) node[above]{$y$};
%   \draw[line width=2pt,blue,-stealth](0,0)--(1,1)
%         node[anchor=south west]{$\boldsymbol{u}$};
%   \draw[line width=2pt,red,-stealth](0,0)--(-1,-1)
%         node[anchor=north east]{$\boldsymbol{-u}$};
% \end{tikzpicture}
% \end{center}
% \caption{\label{fig:vectors} Text size inside figure should be as big as
%   caption's text. Text size inside figure should be as big as
%   caption's text. Text size inside figure should be as big as
%   caption's text. Text size inside figure should be as big as
%   caption's text. Text size inside figure should be as big as
%   caption's text. }
% \end{figure}
% %% %---------------------------


% Here's a typical reference to a floating figure:
% Figure~\ref{fig:vectors}. Floats should usually be placed where latex
% wants then. Figure\ref{fig:vectors} is centered, and has a caption
% that instructs you to make sure that the size of the text within the
% figures that you use is as big as (or bigger than) the size of the
% text in the caption of the figures. Please do. Really.

% In our case, we've explicitly drawn the figure inlined in latex, to
% allow this tex file to cleanly compile. But usually, your figures will
% reside in some file.pdf, and you'd include them in your document
% with, say, \textbackslash{}includegraphics.

% Lists are sometimes quite handy. If you want to itemize things, feel
% free:

% \begin{description}
  
% \item[fread] a function that reads from a \texttt{stream} into the
%   array \texttt{ptr} at most \texttt{nobj} objects of size
%   \texttt{size}, returning returns the number of objects read.

% \item[Fred] a person's name, e.g., there once was a dude named Fred
%   who separated usenix.sty from this file to allow for easy
%   inclusion.
% \end{description}

% \noindent
% The noindent at the start of this paragraph in its tex version makes
% it clear that it's a continuation of the preceding paragraph, as
% opposed to a new paragraph in its own right.


% \subsection{LaTeX-ing Your TeX File}
% %-----------------------------------

% People often use \texttt{pdflatex} these days for creating pdf-s from
% tex files via the shell. And \texttt{bibtex}, of course. Works for us.

% %-------------------------------------------------------------------------------
% \section*{Acknowledgments}
% %-------------------------------------------------------------------------------

% The USENIX latex style is old and very tired, which is why
% there's no \textbackslash{}acks command for you to use when
% acknowledging. Sorry.

% %-------------------------------------------------------------------------------
% \section*{Availability}
% %-------------------------------------------------------------------------------

% USENIX program committees give extra points to submissions that are
% backed by artifacts that are publicly available. If you made your code
% or data available, it's worth mentioning this fact in a dedicated
% section.

% %-------------------------------------------------------------------------------
\bibliographystyle{plain}
\bibliography{reference.bib}
% \bibliography{\jobname}

%%%%%%%%%%%%%%%%%%%%%%%%%%%%%%%%%%%%%%%%%%%%%%%%%%%%%%%%%%%%%%%%%%%%%%%%%%%%%%%%
\end{document}
%%%%%%%%%%%%%%%%%%%%%%%%%%%%%%%%%%%%%%%%%%%%%%%%%%%%%%%%%%%%%%%%%%%%%%%%%%%%%%%%

%%  LocalWords:  endnotes includegraphics fread ptr nobj noindent
%%  LocalWords:  pdflatex acks
